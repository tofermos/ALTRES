% Options for packages loaded elsewhere
\PassOptionsToPackage{unicode}{hyperref}
\PassOptionsToPackage{hyphens}{url}
%
\documentclass[
  a4paper,
]{article}
\usepackage{amsmath,amssymb}
\usepackage{setspace}
\usepackage{iftex}
\ifPDFTeX
  \usepackage[T1]{fontenc}
  \usepackage[utf8]{inputenc}
  \usepackage{textcomp} % provide euro and other symbols
\else % if luatex or xetex
  \usepackage{unicode-math} % this also loads fontspec
  \defaultfontfeatures{Scale=MatchLowercase}
  \defaultfontfeatures[\rmfamily]{Ligatures=TeX,Scale=1}
\fi
\usepackage{lmodern}
\ifPDFTeX\else
  % xetex/luatex font selection
\fi
% Use upquote if available, for straight quotes in verbatim environments
\IfFileExists{upquote.sty}{\usepackage{upquote}}{}
\IfFileExists{microtype.sty}{% use microtype if available
  \usepackage[]{microtype}
  \UseMicrotypeSet[protrusion]{basicmath} % disable protrusion for tt fonts
}{}
\makeatletter
\@ifundefined{KOMAClassName}{% if non-KOMA class
  \IfFileExists{parskip.sty}{%
    \usepackage{parskip}
  }{% else
    \setlength{\parindent}{0pt}
    \setlength{\parskip}{6pt plus 2pt minus 1pt}}
}{% if KOMA class
  \KOMAoptions{parskip=half}}
\makeatother
\usepackage{xcolor}
\usepackage[margin=1in]{geometry}
\usepackage{graphicx}
\makeatletter
\def\maxwidth{\ifdim\Gin@nat@width>\linewidth\linewidth\else\Gin@nat@width\fi}
\def\maxheight{\ifdim\Gin@nat@height>\textheight\textheight\else\Gin@nat@height\fi}
\makeatother
% Scale images if necessary, so that they will not overflow the page
% margins by default, and it is still possible to overwrite the defaults
% using explicit options in \includegraphics[width, height, ...]{}
\setkeys{Gin}{width=\maxwidth,height=\maxheight,keepaspectratio}
% Set default figure placement to htbp
\makeatletter
\def\fps@figure{htbp}
\makeatother
\setlength{\emergencystretch}{3em} % prevent overfull lines
\providecommand{\tightlist}{%
  \setlength{\itemsep}{0pt}\setlength{\parskip}{0pt}}
\setcounter{secnumdepth}{-\maxdimen} % remove section numbering
\ifLuaTeX
\usepackage[bidi=basic]{babel}
\else
\usepackage[bidi=default]{babel}
\fi
\babelprovide[main,import]{catalan}
% get rid of language-specific shorthands (see #6817):
\let\LanguageShortHands\languageshorthands
\def\languageshorthands#1{}
\ifLuaTeX
  \usepackage{selnolig}  % disable illegal ligatures
\fi
\usepackage{bookmark}
\IfFileExists{xurl.sty}{\usepackage{xurl}}{} % add URL line breaks if available
\urlstyle{same}
\hypersetup{
  pdftitle={GUIA RÀPIDA ECOSIA.ORG},
  pdfauthor={@tofermos 2025},
  pdflang={ca-ES},
  hidelinks,
  pdfcreator={LaTeX via pandoc}}

\title{GUIA RÀPIDA \emph{ECOSIA.ORG}}
\usepackage{etoolbox}
\makeatletter
\providecommand{\subtitle}[1]{% add subtitle to \maketitle
  \apptocmd{\@title}{\par {\large #1 \par}}{}{}
}
\makeatother
\subtitle{Caràcterístiques d'Ecosia i configuració per l'ús en Firefox,
Chrome i Edge}
\author{@tofermos 2025}
\date{}

\begin{document}
\maketitle

{
\setcounter{tocdepth}{2}
\tableofcontents
}
\setstretch{1.5}
\newpage
\renewcommand\tablename{Tabla}

\section{1 🌱 El per què d'usar
Ecosia}\label{el-per-quuxe8-dusar-ecosia}

Ecosia és un \textbf{motor de cerca} que utilitza els ingressos
publicitaris per finançar la plantació d'arbres arreu del món. Funciona
de manera similar a Google, però està fortament enfocat en la
\textbf{sostenibilitat i la privacitat} dels usuaris.

\subsection{1.1 Per raons mediamientals}\label{per-raons-mediamientals}

L'aposta per la sostenibilitat i l'impacte positiu en el medi ambient
d'este motor es pot detallar en els següents punts:

\subsubsection{Plantació d'arbres}\label{plantaciuxf3-darbres}

\begin{itemize}
\tightlist
\item
  Aproximadament cada 45 cerques generen suficients ingressos per
  plantar un arbre.
\end{itemize}

\subsubsection{Ús d'energies
renovables}\label{uxfas-denergies-renovables}

\begin{itemize}
\tightlist
\item
  Els seus servidors funcionen amb energia renovable. Fins i tot
  produeixen més electricitat neta de la que consumeixen.
\end{itemize}

\subsubsection{Reducció de la petjada de
carboni}\label{reducciuxf3-de-la-petjada-de-carboni}

\begin{itemize}
\tightlist
\item
  Mentre Ecosia treballa activament per compensar les emissions de CO₂.
\end{itemize}

\subsection{1.2 Per raons ètiques}\label{per-raons-uxe8tiques}

\subsubsection{Privadesa i protecció de dades personals o
organitzacionals}\label{privadesa-i-protecciuxf3-de-dades-personals-o-organitzacionals}

\begin{itemize}
\tightlist
\item
  No emmagatzema cerques de manera permanent.
\item
  No ven dades a anunciants ni utilitza eines de seguiment invasives.
\item
  Les búsquedes són encriptades per protegir la privadesa dels usuaris.
\end{itemize}

\subsubsection{Transparència
financera}\label{transparuxe8ncia-financera}

\begin{itemize}
\tightlist
\item
  Ecosia publica informes mensuals sobre els seus ingressos i com es
  distribueixen els diners, cosa que ofereix més transparència que
  Google.
\end{itemize}

\href{https://ecosia.helpscoutdocs.com/article/402-reports-transparency}{Informes
financers}

\subsubsection{Suport a comunitats
locals}\label{suport-a-comunitats-locals}

\begin{itemize}
\tightlist
\item
  Els projectes de reforestació d'Ecosia tenen un impacte directe en
  comunitats afectades per la desforestació, ajudant a crear llocs de
  treball i restaurar ecosistemes.
\end{itemize}

\newpage

\section{2 🔧 Configuració en
navegadors}\label{configuraciuxf3-en-navegadors}

En les búsquedes, els navegadors (Firefox, Chrome, Edge\ldots) usen un
motor de cerca. L'objectiu és assegurar que usen el \textbf{motor de
cerca d'Ecosia}.

\begin{itemize}
\tightlist
\item
  Quan usem el buscador www.ecosia.org, el navegador usa el motor
  d'Ecosia per a això establirem esta adreça com a \textbf{pàgina
  d'inici}
\item
  No obstant, també establirem com \textbf{motor per defecte} al
  navegador, el d'Ecosia.
\end{itemize}

Vegem estos dos canvis en cada navegador: Firefox, Chrome i Edge.

\subsection{2.1 Navegador Firefox}\label{navegador-firefox}

Entrem la configuració del navegador:
\textbf{\textbar\textbar\textbar{}} \textbf{Paràmetres}

\begin{figure}
\centering
\includegraphics{png/1-Firefox-Parametres.png}
\caption{\emph{Imatge 1: Opció paràmatres de Firefox}}
\end{figure}

\begin{center}\rule{0.5\linewidth}{0.5pt}\end{center}

\subsubsection{Motor de cerca
predeterminat}\label{motor-de-cerca-predeterminat}

Una vegada estem en paràmetres, busquem ``motor'' i seleccionem el que
volem: Ecosia.

\begin{figure}
\centering
\includegraphics{png/2-Firefox-Motor.png}
\caption{\emph{Imatge 2: Motor de búsqueda en Firefox}}
\end{figure}

\begin{center}\rule{0.5\linewidth}{0.5pt}\end{center}

\subsubsection{Pàgina d'inici}\label{puxe0gina-dinici}

A banda del motor podem canviar la pàgina d'inici predeterminada.

Des de l mateixa opció de configuració \emph{paràmetres}, ara busquem
``Inici'' i escrivim la pàgina de ecosia.org

\begin{figure}
\centering
\includegraphics{png/3-Firefox-Inici.png}
\caption{\emph{Imatge 3: Pàgina d'inici en Firefox}}
\end{figure}

\begin{center}\rule{0.5\linewidth}{0.5pt}\end{center}

\subsection{2.2 Navegador Chrome}\label{navegador-chrome}

Entrem la configuració del navegador. Punxat en \textbf{···} i despreś
\textbf{Configuració}

\begin{figure}
\centering
\includegraphics{png/1-Chrome-Configuracio.png}
\caption{\emph{Imatge 4: Opció configuració de Chrome}}
\end{figure}

\begin{center}\rule{0.5\linewidth}{0.5pt}\end{center}

\subsubsection{Motor de cerca
predeterminat}\label{motor-de-cerca-predeterminat-1}

Una vegada estem en la configuració del navegador, busquem ``motor'' i
seleccionem el que volem: ecosia.

\begin{figure}
\centering
\includegraphics{png/2-Chrome-Motor.png}
\caption{\emph{Imatge 5: Motor de búsqueda en Chrome}}
\end{figure}

\begin{center}\rule{0.5\linewidth}{0.5pt}\end{center}

\subsubsection{Pàgina d'inici}\label{puxe0gina-dinici-1}

A banda del motor, podem canviar la pàgina d'inici predeterminada.

Des de la mateixa configuració, seleccionem ``Inici'' i escrivim la
pàgina de ecosia.org

\begin{figure}
\centering
\includegraphics{png/3-Chrome-Inici.png}
\caption{\emph{Imatge 6 Pàgina d'inici en Chrome}}
\end{figure}

\begin{center}\rule{0.5\linewidth}{0.5pt}\end{center}

\subsection{2.3 Navegador Edge (MS
Windows)}\label{navegador-edge-ms-windows}

Punxem en els \textbf{···} i seleccionem \textbf{Configuració}

\begin{figure}
\centering
\includegraphics{png/0-EdgeConfiguracio.png}
\caption{\emph{Imatge 7: Configuració de Edge}}
\end{figure}

\begin{center}\rule{0.5\linewidth}{0.5pt}\end{center}

\subsubsection{Motor de cerca
predeterminat}\label{motor-de-cerca-predeterminat-2}

\begin{figure}
\centering
\includegraphics{png/1-EdgeMotor.png}
\caption{\emph{Imatge 8: Motor de cerca Ecosia en Edge}}
\end{figure}

\subsubsection{Pàgina d'inici}\label{puxe0gina-dinici-2}

\begin{figure}
\centering
\includegraphics{png/2-EdgeInici.png}
\caption{\emph{Imatge 9: Pàgina d'inici}}
\end{figure}

\begin{center}\rule{0.5\linewidth}{0.5pt}\end{center}

\newpage

\section{3 🧩 Extensió d'Ecosia}\label{extensiuxf3-decosia}

L'extensió \textbf{NO és necessària} per treballar amb Ecosia però
ofereix:

\begin{itemize}
\tightlist
\item
  Comptador d'arbres plantats.
\item
  Configuració més ràpida i fàcil.
\item
  Bloqueig de rastrejadors per més privacitat.
\end{itemize}

\subsection{3.1 Extensions en Firefox}\label{extensions-en-firefox}

\begin{figure}
\centering
\includegraphics{png/0-ExtensionsFirefox.png}
\caption{\emph{Imatge 10: Selecciona opció Complements i temes}}
\end{figure}

\begin{center}\rule{0.5\linewidth}{0.5pt}\end{center}

\begin{figure}
\centering
\includegraphics{png/1-ExtensionsFirefox.png}
\caption{\emph{Imatge 11: Buscar l'extensió}}
\end{figure}

\begin{center}\rule{0.5\linewidth}{0.5pt}\end{center}

\subsection{3.2 Extensions en Chrome}\label{extensions-en-chrome}

Entrem en \textbf{···} i seleccionem \textbf{Extensions}.

\begin{figure}
\centering
\includegraphics{png/1-ChromeExtensions.png}
\caption{\emph{Imatge 12: Selecciona Extensions}}
\end{figure}

\begin{center}\rule{0.5\linewidth}{0.5pt}\end{center}

\begin{figure}
\centering
\includegraphics{png/2-ChromeExtensions.png}
\caption{\emph{Imatge 13: Selecciona Chrome Web Store}}
\end{figure}

\begin{center}\rule{0.5\linewidth}{0.5pt}\end{center}

\begin{figure}
\centering
\includegraphics{png/3-ChromeExtensions.png}
\caption{\emph{Imatge 14: Busca i instal·la l'extensió}}
\end{figure}

\begin{center}\rule{0.5\linewidth}{0.5pt}\end{center}

\subsection{3.2 Extensions en Edge}\label{extensions-en-edge}

Entrem en \textbf{···} i seleccionem \textbf{Configuración}. Després
\textbf{Obtenir extensions per a MS Edge}

\begin{figure}
\centering
\includegraphics{png/0-EdgeExtensions.png}
\caption{\emph{Imatge 15: Extensions de Edge}}
\end{figure}

\begin{center}\rule{0.5\linewidth}{0.5pt}\end{center}

\begin{figure}
\centering
\includegraphics{png/1-EdgeExtensions.png}
\caption{\emph{Imatge 16: Instal·lem l'extensió}}
\end{figure}

\begin{center}\rule{0.5\linewidth}{0.5pt}\end{center}

\end{document}
